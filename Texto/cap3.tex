\chapter{Linguagens Regulares: Modelos Alternativos}\label{cap:ExpressionsGrammars}

\epigraph{``As far as I am aware this pronunciation is incorrect in all known languages. ''}{Kenneth Kleene, falando sobre a pronuncia do último no de seu Pai, Stephen Kleene.}

No Capítulo \ref{cap:Automata} foi apresentado a classe das linguagens regulares $\mathcal{L}_{Reg}$ através da formalização dos autômatos finitos (em diversas classes distintas), entretanto, essa não é a única forma de apresentar e formalizar a classe de linguagens $\mathcal{L}_{Reg}$.  Neste capítulo serão apresentados dois modelos alternativos para tal classe de linguagens, sendo estes modelos, o modelo denotacional conhecido por expressões regulares \cite{menezes1998LFA}, e o modelo de gramática geradora \cite{chomsky1956}.

\section{Gramática Regulares}\label{sec:GramaticaRegular}

%Nas seções anteriores foi apresentado o formalismo denotacional (as expressões regulares), e mostrado sua equivalência com o formalismo mecânico (os autômatos). 
Nesta seção será apresentado um terceiro formalismo para as linguagens regulares, sendo este um formalismo gerador (ou axiomático) \cite{menezes1998LFA}. 

\begin{definicao}\label{def:GramaticaLinear}
	Uma gramática formal $G = \langle V, \Sigma, S, P \rangle$ e dita Linear à Direta (à Esquerda), ou simplesmente GLD (GLE), se todas as suas produções são da forma, $A \rightarrow wB$ ($A \rightarrow Bw$), onde $A \in V, B \in V \cup \{\lambda\}$ e $w \in \Sigma^*$.
\end{definicao}

\begin{exemplo}
	A gramática $G_1 = \langle \{B, S , A\}, \{a, b\}, S, P \rangle$ onde $P$ é formado pelas regras:
	\begin{eqnarray*}
		S & \rightarrow& aaB\\
		B & \rightarrow& bb
	\end{eqnarray*}
	é uma GLD.
\end{exemplo}

\begin{exemplo}\label{exe:GLE1}
	A gramática $G_1 = \langle \{X, S , Y\}, \{0, 1\}, S, P \rangle$ onde $P$ é formado pelas regras:
	\begin{eqnarray*}
		S & \rightarrow& X001\\
		S & \rightarrow& Y011\\
		X & \rightarrow& S01\\
		Y & \rightarrow& \lambda
	\end{eqnarray*}
	é uma GLE.
\end{exemplo}

Em uma gramática formal $G$ quando para as sentenças $w, w_1, \cdots, w_n \in (V \cup \Sigma)^*$, existem para todo $1 \leq i \leq n$ a regras $w \rightarrow w_i \in P$, é comum para simplificar a escrita do conjunto de regras usar a notação $w \rightarrow w_1 \mid \cdots \mid w_n$, em vez de escrever cada regra em separado. 

\begin{exemplo}\label{exe:GLE}
	Considere a GLE apresentada no Exemplo \ref{exe:GLE1} o conjunto $P$ da mesma poderia ser escrito como:
	\begin{eqnarray*}
		S & \rightarrow& X001 \mid Y011\\
		X & \rightarrow& S01\\
		Y & \rightarrow& \lambda
	\end{eqnarray*}
\end{exemplo}


Um tipo mais rigoroso de gramática, são as chamadas gramática regulares, que como comentado em \cite{linz2006},  podem ser vista como gramáticas lineares em que as regras são capazes de produzir no máximo único símbolo terminal a cada derivação.

\begin{definicao}[Gramáticas Regulares]
	Uma gramática regular $G$ se ela é linear à esquerda (ou à direita) e toda produção é da forma $A \rightarrow Bw$ $(A \rightarrow wB)$ com $A \in V, B \in V \cup \{\lambda\}$ e $w \in \Sigma \cup \{\lambda\}$.
\end{definicao}

\begin{exemplo}\label{exe:Regular1}
  A estrutura $G = \langle \{A, B, C, D, S\}, \{0, 1\}, S, P \rangle$ com $P$ formado pelas regras:
  \begin{eqnarray*}
		S & \rightarrow& 0A \mid 1B \mid \lambda \\
		A & \rightarrow& 1S \mid 1C\\
		B & \rightarrow& 0S \mid 0D\\
		C & \rightarrow& 0S \mid \lambda\\
		D & \rightarrow& 1S \mid \lambda
	\end{eqnarray*}
  é uma gramática regular.
\end{exemplo}

De forma natural duas gramática $G_1$ e $G_2$ serão ditas equivalentes sempre que elas gerarem a mesma linguagens (volte a Definição \ref{def:LinaugemGramatica} se necessário for para relembrar). O próximo resultado estabelece que gramática lineares (em um única direção) e gramática regulares tem o mesmo poder de geração de linguagens.

\begin{teorema}\label{teo:SimplificacaoRegular}
	$L = \mathcal{L}(G)$ para alguma gramática linear (esquerda ou direita)  $G$ se, e somente se, existe uma gramática regular $G'$ na mesma direção (esquerda ou direita) tal que $L = \mathcal{L}(G')$.
\end{teorema}

\begin{prova}
  $(\Rightarrow)$ Suponha que $L = \mathcal{L}(G)$ para alguma gramática $G = \langle V, \Sigma, S, P \rangle$ tal que $G$ seja linear à esquerda (a prova é similar para o caso à direita). Agora construa uma nova gramática $G' = \langle V', \Sigma, S, P' \rangle$ tal que $P'$ é definido usando as seguintes regras: 
		
	\begin{itemize}
		\item[R1:] Se $A \rightarrow Bw \in P$ onde $B \in V \cup \{\lambda\},  |w| \leq 1$, então $A \rightarrow Bw \in P'$.
		\item[R2:] Se $A \rightarrow Ca_1\cdots a_n \in P$ onde $B \in V$ e $a_i \in \Sigma$ sendo $1 \leq i \leq n$ e $n > 1$, então tem-se que $A \rightarrow B_na_n, B_n \rightarrow B_{n-1}a_{n-1}, \cdots, B_2 \rightarrow Ca_1 \in P'$.
		\item[R3:] Se $A \rightarrow a_1\cdots a_n \in P$ onde $a_i \in \Sigma$ sendo $1 \leq i \leq n$ e $n \geq 2$, então tem-se que $A \rightarrow B_na_n, B_n \rightarrow B_{n-1}a_{n-1}, \cdots, B_2 \rightarrow B_1a_1, B_1 \rightarrow \lambda \in P'$.
	\end{itemize}

	Para as regras R2 e R3 todo $B_i$ é uma nova variável existente em $V'$ que não existe originalmente em $V$. Claramente a gramática $G'$ é regular unitária à esquerda. Também não é difícil mostra por indução sobre o tamanho das derivações que para todo $w \in \Sigma^*$ tem-se que $S \vdash^*_G w$ se, e somente se, $S \vdash^*_{G'} w$, portanto, $\mathcal{L}(G) = \mathcal{L}(G')$.

	$(\Leftarrow)$ Trivial uma vez que toda gramática regular à esquerda (ou à direita) é um caso particular de gramática linear à esquerda (ou à direita).
\end{prova}

\begin{exemplo}\label{exe:GRU}
	Considerando a gramática linear do Exemplo \ref{exe:GLE} usando o Teorema \ref{teo:SimplificacaoRegular} é gerado a gramática regular unitária, 
	$$G' = \langle \{S, B_3, B_2, C_3, C_2, X, D_2, Y\}, \{0, 1\}, S, P'\rangle$$ 
	onde $P'$ é formado pelas regras:
	\begin{eqnarray*}
		S & \rightarrow & B_31 \mid C_31\\
		B_3 & \rightarrow & B_20\\
		B_2 & \rightarrow & X0\\
		C_3 & \rightarrow & C_21\\
		C_2 & \rightarrow & Y0\\
		X & \rightarrow & D_21\\
		D_2 & \rightarrow& S0\\
		Y & \rightarrow & \lambda
	\end{eqnarray*}
\end{exemplo}

Obviamente a gramática do Exemplo \ref{exe:GRU} poderia ser otimizada para usar menos variáveis, porém otimização não é o foco de interesse no Teorema \ref{teo:SimplificacaoRegular}. Os próximos resultados estabelecem o poder de geração das gramáticas regulares à direta.

\begin{lema}\label{lema:SimplificacaoRegular}
	Se $G' = \langle V, \Sigma, S, P\rangle$ é uma gramática regular à direita, então existe uma gramática regular à direita $G' = \langle V', \Sigma, S, P'\rangle$ e toda regra em $G'$ são das formas $A \rightarrow aB$ ou $A \rightarrow \lambda$ com $a \in \Sigma \cup \{\lambda\}$ e $A, B \in V$.
\end{lema}

\begin{prova}
	Suponha que $L = \mathcal{L}(G)$ para alguma gramática regular à direta $G = \langle V, \Sigma, S, P \rangle$, agora construa uma nova gramática $G' = \langle V \cup \{T\}, \Sigma, S, P'\rangle$ com $T \notin V$ e $P'$ sendo construído pelas seguintes regras:
	\begin{itemize}
		\item[$(r_a)$] Se  $A \rightarrow aB \in P$ com  $a \in \Sigma \cup \{\lambda\}$ e $A, B \in V$, então $A \rightarrow aB \in P'$.
		\item[$(r_b)$] Se  $A \rightarrow a \in P$ com  $A\in V$ e $a \in \Sigma$, então $A \rightarrow aT, T \rightarrow \lambda \in P'$.
	\end{itemize}
	note que as regras $r_a$ e $r_b$ garante que não existem transições da forma $A \rightarrow a$ em $P'$ quando $a \in \Sigma$. Agora por indução sobre o tamanho das formas sentenciais, será mostrado que $G'$ preserva todas as derivações de $G$.
	\begin{itemize}
		\item \textbf{(B)}ase da indução: Quando $S \vdash_{G} w$ em uma única derivação, três casos podem ocorrer:
		\begin{itemize}
			\item[(a)] Quando $w = \lambda$ claramente foi aplicada uma regra da forma $S \rightarrow \lambda \in P$, e pela contrução de $P'$ claramente $S \rightarrow w \in P'$, assim $S \vdash_{G'} w$.
			\item[(b)] Quando $w = a$ claramente foi aplicada uma regra da forma $S \rightarrow a \in P$, e pela contrução de $P'$ tem-se que $S \rightarrow aT, T \rightarrow \lambda \in P'$, consequentemente, $S \vdash_{G'}^* w$.
			\item[(c)]  Quando $w = aB$ claramente foi aplicada uma regra da forma $S \rightarrow aB \in P$, e pela contrução de $P'$ tem-se que $S \rightarrow aB \in P'$, e portanto, $S \vdash_{G'} w$.
		\end{itemize}
		\item \textbf{(H)}ipótese indutiva: Suponha que $S \vdash_{G}^* a_1\cdots a_{n-1}A_n$ com $n \geq 1$ derivações e $S \vdash_{G'}^* a_1\cdots a_{n-1}A_n$ com $n$ ou $n+1$ derivações.
		\item \textbf{(P)}asso indutivo: Seja $S \vdash_{G}^* a_1\cdots a_{n-1}a_n$ com $n + 1$ derivações, logo existe uma forma sentencial $a_1\cdots a_{n-1}A_n$ derivada em $G$ com $n$ derivações e $A_n \rightarrow a \in P$ com $a \in \Sigma \cup \{\lambda\}$. Agora por \textbf{(H)}, existe $a_1\cdots a_{n-1}A_n$ é derivada em $G$ com $n$ derivações e em $G'$ com $n$ ou $n+1$ derivações. Como $S \vdash_{G}^* a_1\cdots a_{n-1}a_n$ com $n + 1$ derivações isso implica que $A_n \rightarrow a_n \in P$ com $a_n \in \Sigma \cup \{\lambda\}$, logo quando $a_n = \lambda$ então $A_n \rightarrow a_n \in P'$ e, portanto,  $S \vdash_{G'}^* a_1\cdots a_{n-1}a_n$ com $n + 1$ derivações, por outro lado, quando $a_n \in \Sigma$ tem-se que $A_n \rightarrow a_nT, T \rightarrow \lambda \in P'$, e consequentemente,  $S \vdash_{G'}^* a_1\cdots a_{n-1}a_n$ com $n + 2$ derivações.
	\end{itemize}
	Portanto, o raciocínio indutivo anterior garante que $G'$ preserva todas as derivações de $G$, logo quando $w \in \mathcal{L}(G)$ tem-se que $w \in \mathcal{L}(G')$, consequentemente,  $ \mathcal{L}(G) \subset \mathcal{L}(G')$. Por outro lado, pelo construção de $G'$ tem-se sem perda de generalidade\sidefootnote{O caso $a_1\cdots a_n = \lambda$ é trivial e não será demonstrado aqui.} que $a_1\cdots a_n \neq \lambda$ é tal que,
	\begin{eqnarray*}
		a_1\cdots a_n \in \mathcal{L}(G') & \Rightarrow &  (\exists A_{n} \in V')[S \vdash_{G'}^* a_1\cdots a_{n}A_n \land A_n \rightarrow \lambda \in P']\\
		& \Rightarrow & (\exists A_1, \cdots, A_{n} \in V')[S \rightarrow a_1A_1, \\
		& & \cdots, A_{n-1} \rightarrow a_nA_n, A_n \rightarrow \lambda \in P']\\
		& \Rightarrow & (\exists A_1, \cdots, A_{n} \in V)[S \rightarrow a_1A_1,\\
		& &  \cdots, A_{n-1} \rightarrow a_nA_n, A_n \rightarrow \lambda \in P]\\
		& & \lor  (\exists A_1, \cdots, A_{n-1} \in V)[S \rightarrow a_1A_1, \cdots, A_{n-1} \rightarrow a_{n} \in P]\\
		& \Rightarrow &  (\exists A_{n} \in V)[S \vdash_{G}^* a_1\cdots a_{n}A_n \land A_n \rightarrow \lambda \in P]\\
		& & \lor (\exists A_{n-1} \in V)[S \vdash_{G}^* a_1\cdots a_{n-1}A_n \land A_n \rightarrow a_n \in P]\\
		& \Rightarrow & a_1\cdots a_n \in \mathcal{L}(G) \lor a_1\cdots a_n \in \mathcal{L}(G)\\
		& \Rightarrow & a_1\cdots a_n \in \mathcal{L}(G)
	\end{eqnarray*}
	E portanto, $\mathcal{L}(G') \subset \mathcal{L}(G)$. Agora desde que $ \mathcal{L}(G) \subset \mathcal{L}(G')$ e $\mathcal{L}(G') \subset \mathcal{L}(G)$, tem-se que $\mathcal{L}(G') = \mathcal{L}(G)$, o que completa a prova.
\end{prova}

\begin{teorema}\label{teo:GRD-AFD}
	$L = \mathcal{L}(G)$ para alguma gramática regular à direta $G$ se, e somente se, $L$ é uma linguagem regular.
\end{teorema}

\begin{prova}
	$(\Rightarrow)$ Suponha que $L = \mathcal{L}(G')$ para alguma gramática regular à direta $G' = \langle V, \Sigma, S, P \rangle$ tal que $L = \mathcal{L}(G')$, sem perda de generalidade, pelo Lema \ref{lema:SimplificacaoRegular} pode-se assumir que toda regra em $P$ é da forma $A \rightarrow aB$ ou $A \rightarrow \lambda$ com $A, B \in V$ e $a \in \Sigma \cup \{\lambda\}$, dito isto, pode-se agora construir um $\lambda$-AFN $M = \langle V, \Sigma, \underline{\delta_N}, S, F \rangle$ tal que:
	\begin{eqnarray*}
		B \in \underline{\delta_N}(A, a) & \Longleftrightarrow & A \rightarrow aB \in P
	\end{eqnarray*}
	como $A, B \in V$ e $a \in \Sigma \cup \{\lambda\}$ e $F = \{A \in V \mid A \rightarrow \lambda \in P\}$. Agora será mostrado por indução sobre o tamanho das derivações em $G'$ que se $w$ é derivada por $G'$ e $w \in \Sigma^*$, então é aceita por $M$.
	\begin{itemize}
		\item \textbf{(B)}ase da indução: Quando $w$ é derivada em $G'$ com uma única derivação tem-se então duas situações possíveis:
		\begin{itemize}
			\item[(1)] Quando $w = \lambda$, obrigatoriamente existe uma regra da forma $S \rightarrow \lambda$, e pela construção de $M$ tem-se que $S \in \underline{\delta_N}(S, \lambda)$, logo $\widehat{\underline{\delta_N}}(S, \lambda) \cap F \neq \emptyset$ e, portanto, $\lambda \in L(M)$. 
			\item[(2)] Quando $w = aB$, existe em $P$ uma regra da forma $S \rightarrow aB$ com $a \in \Sigma \cup \{\lambda\}$ e $B \in V$, assim pela construção de $M$ tem-se que $B \in \underline{\delta_N}(A, a)$. Como $aB \notin \Sigma^*$ não há mais nada a fazer nesse caso.
		\end{itemize}
		
		\item \textbf{(H)}ipótese indutiva: Suponha que $S \vdash^*_{G'} w$ em $n$ derivação com $n \geq 1$ tal que:
		\begin{itemize}
			\item[(1)] Se $w \in \Sigma^*$, então $w \in \mathcal{L}(M)$.
			\item[(2)] Se $w = a_1\cdots a_{n-1}B$ com $a_i \in \Sigma \cup \{\lambda\}$ para todo $1 \leq i \leq n-1$ e $B \in V$, então $B \in \widehat{\underline{\delta_N}}(S, a_1\cdots a_{n-1})$.
		\end{itemize}
		
		\item \textbf{(P)}asso indutivo: Agora dado que $S \vdash^*_{G'} w'$ em $n+1$ derivações, tem-se obrigatoriamente que acontece o caso (2) de \textbf{(HI)} e nesse caso duas situações são possíveis:
		\begin{itemize}
			\item[(1)] Se $w' \in \Sigma^*$, então $w = a_1\cdots a_{n-1}B$ com $a_i \in \Sigma \cup \{\lambda\}$ para todo $1 \leq i \leq n-1$ e existe em $P$ uma produção $B \rightarrow \lambda$, e assim, $w' = a_1\cdots a_{n-1}$, nesta situação pelo caso (2) de \textbf{(HI)} tem-se que $B \in \widehat{\underline{\delta_N}}(S, a_1\cdots a_{n-1})$ e como $B \rightarrow \lambda$ pela construção de $M$ tem-se que $B \in F$, consequentemente, $\widehat{\underline{\delta_N}}(S, a_1\cdots a_{n-1}) \cap F \neq \emptyset$ e, portanto, $w' \in \mathcal{L}(M)$.
			\item[(2)] Se $w' = a_1\cdots a_{n-1}B$ com $a_i \in \Sigma \cup \{\lambda\}$ para todo $1 \leq i \leq n-1$ e $B \in V$, então pela construção de $M$ tem-se que $B \in \widehat{\underline{\delta_N}}(S, w)$ como $w' \notin \Sigma^*$ não há mais nada a fazer nesse caso.
		\end{itemize}
	\end{itemize}
	Portanto, o raciocínio indutivo anterior garante que sempre que  $w$ é derivada por $G'$ e $w \in \Sigma^*$ tem-se que $w \in \mathcal{L}(M)$ e assim pode-se afirmar pelo Corolário \ref{col:RegularLAFN} que $L$ é regular. 
	
	$(\Leftarrow)$ Suponha que $L$ é uma linguagem regular assim por definição existe um AFD $M = \langle Q, \Sigma, \delta, q_0, F \rangle$ tal que $L = \mathcal{L}(M)$, assim construa uma gramática regular à direita $G = \langle Q, \Sigma, q_0, P \rangle$ onde o conjunto $P$ é definido usando as regras a seguir, 
	\begin{itemize}
		\item[(a)] Se $\delta(q_i, a) = q_j$, então $q_i \rightarrow aq_j \in P$.
		\item[(b)] Se $q_i \in F$, então $q_i \rightarrow \lambda \in P$.
	\end{itemize} 
	Agora note que para todo $w \in \Sigma^*$ com $w = a_1 \cdots a_n$ tem-se que, 
	\begin{eqnarray*}
		w \in \mathcal{L}(M) & \Longleftrightarrow & \widehat{\delta}(q_0, w) \in F\\
		& \Longleftrightarrow & \widehat{\delta}(q_0, a_1 \cdots a_n) \in F\\
		& \Longleftrightarrow & (\exists q_f \in F)[\widehat{\delta}(q_0, w) = q_f]\\
		& \Longleftrightarrow & (\exists q_1, \cdots, q_{n-1} \in Q,  q_f \in F)\\
		& & [\delta(q_0, a_1) = q_1 \land \cdots \land \delta(q_{n-1}, a_n) = q_f]\\
		& \Longleftrightarrow & (\exists q_1, \cdots, q_{n-1} \in Q,  q_f \in F)\\
		& & [q_0 \rightarrow a_1q_1, \cdots, q_{n-1} \rightarrow a_nq_f, q_f \rightarrow \lambda \in P]\\
		& \Longleftrightarrow & q_0 \vdash^* a_1 \cdots a_n\\
		& \Longleftrightarrow & q_0 \vdash^* w\\
		& \Longleftrightarrow & w \in \mathcal{L}(G)
	\end{eqnarray*} 
	portanto, $\mathcal{L}(M) = \mathcal{L}(G)$ o que conclui a prova.
\end{prova}

\begin{lema}\label{lema:GRD-Reversa}
	Se $L$ é gerada por uma gramática à direita, então $L^r$ é gerada por uma gramática regular à direita.
\end{lema}

\begin{prova}
	Suponha que $L$ é gerada por uma gramática à direita $G$, ou seja, $L = \mathcal{L}(G)$, assim pelo Teorema \ref{teo:GRD-AFD} tem-se que $L$ é regular, logo existe um AFD $M = \langle Q, \Sigma, \delta, q_0, F\rangle$ tal que $L = \mathcal{L}(M)$, agora construa um $\lambda$-AFN $M_1 = \langle Q \cup \{q_f\}, \Sigma, \underline{\delta_N}, q_0, \{q_f\}\rangle$ tal que,
	\begin{eqnarray*}
		\underline{\delta_N}(q, a) & = & \left\{\begin{array}{ll}	
			\{\delta(q, a)\}, & \hbox{se } q \in Q, a \in \Sigma\\
			\{q_f\}, & \hbox{se } q \in F, a = \lambda \\
			\emptyset, & \hbox{qualquer outro caso}
		\end{array}\right.
	\end{eqnarray*}
	claramente $L = \mathcal{L}(M_1)$, agora construa um novo $\lambda$-AFN $M_2 = \langle Q \cup \{q_f\}, \Sigma, \underline{\delta'_N}, q_f, \{q_0\}\rangle$ onde para todo $q \in Q \cup \{q_f\}$ e $a \in \Sigma \cup \{\lambda\}$ tem-se que,
	\begin{eqnarray*}
		q_i \in \underline{\delta'_N}(q_j, a) & \Longleftrightarrow & q_j \in \underline{\delta_N}(q_i, a)
	\end{eqnarray*}
	pela construção de $M_2$ é claro que  $w \in \mathcal{L}(M_1) \Longleftrightarrow w^r \in \mathcal{L}(M_2)$ e, portanto, $L^r = \mathcal{L}(M_2)$. Desde que $M_2$ é um $\lambda$-AFN pelo Corolário \ref{col:RegularLAFN} tem-se que $L^r$ é uma linguagem regular, consequentemente, pelo Teorema \ref{teo:GRD-AFD} existe uma gramática regular à direita $G'$ tal que $L = \mathcal{L}(G')$, o que conclui a prova.
\end{prova}

O próximo resultado mostra que gramática regulares à esquerda e à direita são equivalentes.

\begin{teorema}[Mudança de direção regular]\label{teo:MudacaDeLadoGramatica}
	$L$ é gerada por uma gramática à esquerda se, e somente se, $L$ é gerada por uma gramática regular à direita.
\end{teorema}

\begin{prova}
	$(\Rightarrow)$ Suponha que $L$ é gerada por uma gramática à esquerda $G = \langle V, \Sigma, S, P \rangle$, assim sem perda de generalidade\sidefootnote{Basta gerar uma nova gramática onde toda regra da forma $A \rightarrow a$ com $a \in \Sigma$ foi substituída pelas regras $A \rightarrow Ca$ e $C \rightarrow \lambda$ onde $C$ é uma variável nova criada.} pode-se assumir que todas as regras em $P$ são da forma $A \rightarrow Ba$ com $A \in V, B \in V \cup \{\lambda\}$ e $a \in \Sigma \cup \{\lambda\}$,  agora construa um $\lambda$-AFN $M = \langle V \cup \{q_f\}, \Sigma, \underline{\delta_N}, S, \{q_f\} \rangle$ onde, 
	\begin{eqnarray*}
		B \in \underline{\delta_N}(A, a) & \Longleftrightarrow & A \rightarrow Ba \in P\\
		q_f \in \underline{\delta_N}(A, \lambda)& \Longleftrightarrow & A \rightarrow \lambda \in P
	\end{eqnarray*} 
	Agora note que para todo $w = a_1\cdots a_n \in \Sigma^*$ tem-se que,
	\begin{eqnarray*}
		w \in \mathcal{L}(G') & \Longleftrightarrow &  a_1\cdots a_n \in \mathcal{L}(G')\\
		& \Longleftrightarrow & S \vdash_{G'}^* a_1\cdots a_n\\
		& \Longleftrightarrow & (\exists A_1 \cdots A_n, S \in V)\\
		& & [S \rightarrow A_na_n, A_n \rightarrow A_{n-1}a_{n-1}, \cdots, A_{2} \rightarrow A_1a_1, A_1 \rightarrow \lambda \in P']\\
		& \Longleftrightarrow & (\exists A_1 \cdots A_n, S \in V)\\
		& & [A_n \in \underline{\delta_N}(S, a_n), A_{n-1} \in \underline{\delta_N}(A_n, a_{n-1}), \cdots,  A_{1} \in \underline{\delta_N}(A_2, a_{1}), \\
		& &  q_f \in \underline{\delta_N}(A_1, \lambda)]\\
		& \Longleftrightarrow & a_n\cdots a_1 \in \mathcal{L}(M)\\
		& \Longleftrightarrow & w^r \in \mathcal{L}(M)
	\end{eqnarray*}
	Logo $\mathcal{L}(M) = \mathcal{L}(G)^r$, desde que $M$ é um $\lambda$-AFN tem-se pelo Corolário \ref{col:RegularLAFN} que $\mathcal{L}(G)^r$ é regular, assim pelo Teorema \ref{teo:GRD-AFD} existe uma gramática regular à direita $\hat{G_1}$ que a gera, ou seja, $\mathcal{L}(\hat{G_1}) = \mathcal{L}(G)^r$, mas pelo Lema \ref{lema:GRD-Reversa} irá existir outra gramática regular à direita $\hat{G_2}$ tal que $\mathcal{L}(\hat{G_2}) = \mathcal{L}(\hat{G_1})^r$, mas $ \mathcal{L}(\hat{G_1})^r = (\mathcal{L}(G)^r)^r = (L^r)^r = L$, portanto, $L$ é gerada por uma gramática regular à direita. 
	
	$(\Leftarrow)$ Suponha que que $L$ é gerada por uma gramática regular à direita, ou seja, que existe uma gramática regular a direita $G$ tal que $L = \mathcal{L}(G)$, assim pelo Lema \ref{lema:GRD-Reversa} irá existir outra gramática regular à direita $G_1 = \langle V, \Sigma, S, P \rangle$ tal que $L^r = \mathcal{L}(G_1)$, sem perda de generalidade pode-se assumir todas as suas produções em $G_1$ são da forma $A \rightarrow aB$ com $A \in V, B \in V \cup \{\lambda\}$ e $a \in \Sigma \cup \{\lambda\}$. Dito isso construa uma nova gramática $G_2 = \langle V, \Sigma, S, P'\rangle$ onde $P' = \{A \rightarrow Ba \mid A \rightarrow aB \in P\}$, claramente $G_2$ é regular à esquerda. Mas pela construção de $G_2$ fica claro que $S \vdash_{G_1}^* w \Longleftrightarrow S \vdash_{G_2}^* w^r$, logo $\mathcal{L}(G_1)^r = \mathcal{L}(G_2)$, mas desde que, $\mathcal{L}(G_1)^r = (L^r)^r = L$, tem-se então que $L$ é gerada por uma gramática linear à esquerda, o que completa  a prova.
\end{prova}

\begin{exemplo}
	Dado a gramática regular à direita $G_1 = \langle \{A, B, C\}, \{a, b\}, A, P_1 \rangle$ com $P_1$ é formado pelas seguintes regras,
	\begin{eqnarray*}
		A & \rightarrow & aC \mid B \\
		B & \rightarrow & bB \mid \lambda\\
		C & \rightarrow & aA
	\end{eqnarray*}
	 claramente $\mathcal{L}(G_1) = \{w \in \{a, b\}^* \mid w = a^{2m}b^n \hbox{ com } m,n \in \mathbb{N}\}$,  agora usando a construção exposta pelo Teorema \ref{teo:MudacaDeLadoGramatica}, é possível construir a gramática regular à esquerda $G_2 = \langle \{A, B, C\}, \{a, b\}, B, P_2 \rangle$ onde $P_2$ é formado pelas seguintes regras,
	 \begin{eqnarray*}
	 	B & \rightarrow & Bb \mid Ab \mid A\\
	 	A & \rightarrow & Ca \mid \lambda\\
	 	C & \rightarrow & Aa 
	 \end{eqnarray*}
	 e obviamente $\mathcal{L}(G_2) = \{w \in \{a, b\}^* \mid w = a^{2m}b^n \hbox{ com } m,n \in \mathbb{N}\}$.
\end{exemplo}

\section{Expressões regulares}\label{sec:ExpressionsRegulares}

Até agora as linguagens foram vistas sobre a ótica das máquinas de computação, isto é, sobre a perspectiva dos autômatos finitos, e sobre a ideia de gramática regulares, ou seja, estruturas geradoras. Em tais perspectivas as linguagens são vistas como sendo conjuntos de palavras sobre os quais as máquinas tinha a tarefa de reconhecer seus elementos, ou como elementos que deveriam ser gerados (ou derivados) pelas gramáticas. 

Neste seção será apresentada uma visão alternativa, essa visão tem um aspecto mais algébrico para as linguagens. Tal visão foi introduzida pelo matemático e lógico americano Stephen Kleene em seu seminal \textit{paper} intitulado como, ``\textit{Representation of events in nerve nets and finite automata}'' \cite{kleene1951}. Sobre a perspectiva introduzida por Kleene é importante mencionar que ela consiste de um sistema formal (com sintaxe e semântica), chamamos esse sistema formal de expressões  regulares. Tal sistema tem a caracteristica de ser denotacional, isto é, a linguagem é denotada por uma expressão (em geral simples).

\begin{definicao}[Conjunto das Expressões Regulares (Sintaxe)]\label{def:ExpRegularesSintaxe}
	Seja $\Sigma$ uma alfabeto, o conjunto de todas as expressões regulares sobre $\Sigma$, denotado por $Exp_\Sigma$, é o conjunto indutivamente gerado pelas seguintes regras.
	\begin{itemize}
		\item[ ]\textbf{(B)ase}: $\emptyset, \lambda$ e cada $a \in \Sigma$, são expressões regulares\footnote{As expressões regulares da base costumam ser chamadas de expressões regulares primitivas.}.
		\item[ ]\textbf{(P)asso indutivo}:  Se $r_1, r_2 \in Exp_\Sigma$, então $r_1 + r_2, r_1 \cdot r_2, r_1^*, (r_1) \in Exp_\Sigma$.
		\item[ ]\textbf{(F)echo}: $Exp_\Sigma$ é exatamente o conjunto dos elementos obtidos a partir \textbf{(B)} ou usando-se uma quantidade finita (podendo ser nula) de aplicações de \textbf{(P)}.
	\end{itemize}
\end{definicao}

No que diz respeito as expressões regulares é comum assumir que $+, \cdot, (, ), ^* \notin \Sigma$, assim uma expressão regular nem sempre é uma palavra sobre $\Sigma$, em geral os símbolos $+$ e $\cdot$ são lidos como soma e produto \cite{carroll1989}, e o $^*$ é lido com operador estrela de Kleene, ou simplesmente, estrela. Além disso, como dito em \cite{benjaLivro2010} se $r_1, r_2 \in Exp_\Sigma$, então costuma-se escrever $r_1r_2$ em vez de $r_1 \cdot r_2$.

\begin{exemplo}
	Considerando o alfabeto $\{0,1\}$ tem-se que $(1 + 1)0, 01 \in Exp_\Sigma$. Essa afirmação pode ser verificada construindo a ávore de dedução de tais palavras, e isso é feito abaixo.

	\[
	\begin{prooftree}
		\hypo{ }
		\infer1[\textbf{(B)}]{ 1 }
		\hypo{ }
		\infer1[\textbf{(B)}]{ 1 }
		\infer2[\textbf{(P)}]{ 1 + 1 }
		\infer1[\textbf{(P)}]{ (1 + 1) }
		\hypo{ }
		\infer1[\textbf{(B)}]{ 0 }
		\infer2[\textbf{(P)}]{ (1 + 1)0 }
	\end{prooftree}
	\quad 
	\ \ \ \ \ \ \ \ \ \
	\quad 
	\begin{prooftree}
		\hypo{ }
		\infer1[\textbf{(B)}]{ 0 }
		\hypo{ }
		\infer1[\textbf{(B)}]{ 1 }
		\infer2[\textbf{(P)}]{ 01 }
	\end{prooftree}
	\]
	e isso mostra que de fato $(1 + 1)0, 01 \in Exp_{\{0,1\}}$.
\end{exemplo}

\begin{exemplo}
	Considerando o alfabeto $\{a, b, c\}$ tem-se que:

	\[
		\begin{prooftree}
			\hypo{ }
			\infer1[\textbf{(B)}]{ a }
			\hypo{ }
			\infer1[\textbf{(B)}]{ \emptyset }
			\hypo{ }
			\infer1[\textbf{(B)}]{ b }
			\hypo{ }
			\infer1[\textbf{(B)}]{ c }
			\infer1[\textbf{(P)}]{ c^* }
			\infer2[\textbf{(P)}]{ bc^* }
			\infer1[\textbf{(P)}]{ (bc^*) }
			\infer2[\textbf{(P)}]{ \emptyset + (bc^*) }
			\infer1[\textbf{(P)}]{ (\emptyset + (bc^*)) }
			\infer2[\textbf{(P)}]{ a(\emptyset + (bc^*)) }
		\end{prooftree}
	\]

	logo $a(\emptyset + (bc^*)) \in Exp_{\{a, b, c\}}$.
\end{exemplo}

A seguir será formalizado o conceito de semântica para as expressões regulares, sendo que tal semântica pode ser visto como uma \textbf{semântica denotacional}\sidefootnote{Aquela em que as valoração usadas, são funções que mapeiam palavras da linguagem para funções parciais que representam o comportamento dos programas.} \cite{scott1971}, que apresenta significado as operações de soma e multiplicação. Antes porém, vale ressaltar que, uma expressão regular não é, em geral, uma palavra de alguma linguagem sobre $\Sigma$.

O conceito de semântica para as expressões regulares (descrito a seguir), corresponde formalmente a ideia de uma \textbf{semântica denotacional}\footnote{Uma semântica denotacional é aquela em que as funções de valoração usadas, são funções que mapeiam palavras da linguagem para funções parciais que representam o comportamento dos programas.} \cite{scott1971}. Inicialmente, será apresentado tal ideia, seguido de alguns exemplo e, por fim, será descutido questões associadas a semântica.

\begin{definicao}[Semântica das Expressão Regulares]\label{def:ExpRegularesSemantica}
	Seja $Exp_\Sigma$ o conjunto das expressões regulares sobre $\Sigma$,  a semântica (ou interpretação) de $Exp_\Sigma$ é uma função $\mathcal{L}: Exp_\Sigma \rightarrow \wp(\Sigma^*)$ definida recursivamente para todo $r, r_1, r_2  \in Exp_\Sigma$ pelas seguintes regras.
	\begin{itemize}
		\item[(i)] Se $r \in \Sigma \cup \{\lambda\}$, então $\mathcal{L}(r) = \{r\}$.
		\item[(ii)] Se $r = \emptyset$, então $\mathcal{L}(r) = \emptyset$.
		\item[(iii)] Se $r = r_1 + r_2$, então $\mathcal{L}(r) = \mathcal{L}(r_1) \cup \mathcal{L}(r_2)$.
		\item[(iv)] Se $r = r_1 \cdot r_2$, então $\mathcal{L}(r) = \mathcal{L}(r_1)\mathcal{L}(r_2)$.
		\item[(v)] Se $r = r_1^*$, então $\mathcal{L}(r) = (\mathcal{L}(r_1))^*$.
		\item[(vi)] Se $r = (r_1)$, então $\mathcal{L}(r) = (\mathcal{L}(r_1))$.	
	\end{itemize}
\end{definicao}

\begin{exemplo}\label{exe:ValoracaoExpressao1}
	Dado o alfabeto $\{0,1\}$ e $0^* \in Exp_\Sigma$ tem-se que:
	\begin{eqnarray*}
		\mathcal{L}(0^*) & = & (\mathcal{L}(0))^*\\
		& = & (\{0\})^*\\
		& = & \{\lambda, 00, 0000, 000000, 00000000, \cdots\}
	\end{eqnarray*}
	ou seja, a valoração da expressão regular $0^*$ consiste da linguagem de todas as palavras $w$ sobre o alfabeto $\{0,1\}$ sem nenhum 1.
\end{exemplo}

\section{Coisa que não são expressões regulares!}

Muitos ambientes de computação e linguagens de programação oferecem suporte a padrões chamados \textit{regex} que são consideravelmente mais gerais e poderosos do que as expressões regulares, vistas na seção anterior. Os \textit{regex} oferecidos por linguagens de programação incluem símbolos especiais que representam negação, classes de caracteres (por exemplo, letras maiúsculas ou dígitos), intervalos contíguos de caracteres, limites de linha e de palavra, repetição limitada\sidefootnote{Que ẽ algo oposto totalmente à repetição ilimitada (fecho de Kleene) permitida pelas expressões vistas aqui}, referências a subexpressões anteriores (back-references) e até variáveis locais. 

Então resumidamente e de forma direta, apesar de sua etimologia óbvia, uma regex oferecida por uma linguagem de programação não é necessariamente uma expressão regular, e não necessariamente descreve uma linguagem regular\sidefootnote{Veja a discussão em \url{http://stackoverflow.com/a/1732454/775369}}.

Outro tipo de padrão que frequentemente é confundido com expressões regulares são os \textit{globs}, que são padrões usados na maioria dos programas de \textit{shells} em  sistemas \textit{Unix-like} e em algumas linguagens de script para representar conjuntos de nomes de arquivos. \textit{Globs} incluem símbolos para caracteres arbitrários únicos (?), caracteres únicos de um intervalo específico ([a-z]), substrings arbitrárias (*) e substrings de um conjunto finito especificado ({foo,ba{r,z}}). \textit{Globs} são, na verdade, significativamente menos poderosos do que expressões regulares \cite{erickson2014}.






