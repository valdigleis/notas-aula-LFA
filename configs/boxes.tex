%---------------------------------------------------------------------------------------
%	Ambiente do Teorema
%---------------------------------------------------------------------------------------
\newlength{\spacelength}
\settowidth{\spacelength}{\normalfont\ }
\declaretheoremstyle[
    headfont={\bfseries\sffamily\footnotesize},
    notefont={\normalfont},
    bodyfont={\normalfont},
    headpunct={\relax},%\newline,
    headformat={%
        \makebox[0pt][r]{\NAME\ \NUMBER\hspace{\marginparsep}}\hskip-\spacelength{\normalsize\NOTE}},
]{teorema}

\tcolorboxenvironment{teorema}{
  boxrule=0pt,
  boxsep=0pt,
  colback={White!90!Dandelion},
  enhanced jigsaw, 
  borderline west={1pt}{0pt}{Dandelion},
  sharp corners,
  before skip=10pt,
  after skip=10pt,
  left=5pt,
  right=5pt,
  breakable,
}

\declaretheorem[style=teorema]{teorema}

%---------------------------------------------------------------------------------------
%	Ambiente para demonstração
%---------------------------------------------------------------------------------------
\let\proof\relax
\let\endproof\relax

\declaretheoremstyle[
    headfont={\sffamily\footnotesize},
    notefont={\normalfont},
    bodyfont={\normalfont},
    headpunct={\relax},
    headformat={%
        \makebox[0pt][r]{\NAME\hspace{\marginparsep}}\hskip-\spacelength{\normalsize\NOTE}},
]{prova}

\tcolorboxenvironment{prova}{
  boxrule=0pt,
  boxsep=0pt,
  blanker,
  borderline west={1pt}{0pt}{NavyBlue!80!white},
  before skip=10pt,
  after skip=10pt,
  left=5pt,
  right=5pt,
  breakable,
}

\declaretheorem[
    style=prova,
    qed=\qedsymbol]{prova}

%---------------------------------------------------------------------------------------
%	Ambiente para Intuição
%---------------------------------------------------------------------------------------
\declaretheoremstyle[
    headfont={\footnotesize\itshape},
    notefont={\normalfont},
    bodyfont={\normalfont},
    headpunct={\relax},
    headformat={%
        \makebox[0pt][r]{\NAME\hspace{\marginparsep}}\hskip-\spacelength{\NOTE}},
]{claim}

\declaretheorem[style=claim]{Intuicao}

%---------------------------------------------------------------------------------------
%	Ambiente para Lema
%---------------------------------------------------------------------------------------
\theoremstyle{teorema}
\newtheorem{lema}{Lema}
\tcolorboxenvironment{lema}{
  boxrule=0pt,
  boxsep=0pt,
  colback={White!90!tomato},
  enhanced jigsaw,
  borderline west={1pt}{0pt}{tomato},
  before skip=10pt,
  after skip=10pt,
  sharp corners,
  left=5pt,
  right=5pt,
  breakable,
}

%---------------------------------------------------------------------------------------
%	Ambiente para Corolario
%---------------------------------------------------------------------------------------
\theoremstyle{teorema}
\newtheorem{corolario}{Corolario}
\tcolorboxenvironment{corolario}{
  boxrule=0pt,
  boxsep=0pt,
  colback={White!70!pink},
  enhanced jigsaw,
  borderline west={1pt}{0pt}{pink},
  before skip=10pt,
  after skip=10pt,
  sharp corners,
  left=5pt,
  right=5pt,
  breakable,
}

%---------------------------------------------------------------------------------------
%	Ambiente para Proposições
%---------------------------------------------------------------------------------------
\theoremstyle{teorema}
\newtheorem{proposicao}{Proposição}
\tcolorboxenvironment{proposicao}{
  boxrule=0pt,
  boxsep=0pt,
  colback={White!95!purple},
  enhanced jigsaw,
  borderline west={1pt}{0pt}{purple},
  before skip=10pt,
  after skip=10pt,
  sharp corners,
  left=5pt,
  right=5pt,
  breakable,
}

%---------------------------------------------------------------------------------------
%	Ambiente para Definição
%---------------------------------------------------------------------------------------
\theoremstyle{teorema}
\newtheorem{definicao}{Definição}[chapter]
\tcolorboxenvironment{definicao}{
  boxrule=0pt,
  boxsep=0pt,
  colback={White!90!Cerulean},
  enhanced jigsaw,
  borderline west={1pt}{0pt}{Cerulean},
  sharp corners,
  before skip=10pt,
  after skip=10pt,
  sharp corners,
  left=5pt,
  right=5pt,
  breakable,
}

%---------------------------------------------------------------------------------------
%	Ambiente para Exemplos
%---------------------------------------------------------------------------------------
\declaretheoremstyle[
    headfont={\footnotesize\itshape},
    notefont={\normalfont},
    bodyfont={\normalfont},
    headpunct={\relax},
    headformat={%
        \makebox[0pt][r]{\NAME\ \NUMBER\hspace{\marginparsep}}\hskip-\spacelength{\normalsize\NOTE}},
]{exemplos}

\theoremstyle{exemplos}
\newtheorem{exemplo}{Exemplo}[section]
\tcolorboxenvironment{exemplo}{
  boxrule=0pt,
  boxsep=0pt,
  blanker,
  borderline west={1pt}{0pt}{Black},
  sharp corners,
  before skip=10pt,
  after skip=10pt,
  left=5pt,
  right=5pt,
  breakable,
}

%---------------------------------------------------------------------------------------
%	Ambiente para Atenção
%---------------------------------------------------------------------------------------
\theoremstyle{claim}
\newtheorem{atencao}{{\footnotesize\color{Red}\textdbend} Atenção}
\tcolorboxenvironment{atencao}{
  boxrule=0pt,
  boxsep=0pt,
  blanker,
  borderline west={1pt}{0pt}{Red},
  sharp corners,
  before skip=10pt,
  after skip=10pt,
  left=5pt,
  right=5pt,
  breakable,
}

%---------------------------------------------------------------------------------------
%	Ambiente para Questão
%---------------------------------------------------------------------------------------
\theoremstyle{teorema}
\newtheorem{questao}{Questão}[chapter]
\tcolorboxenvironment{questao}{
  boxrule=0pt,
  boxsep=0pt,
  colback={White!70!pink},
  enhanced jigsaw,
  borderline west={1pt}{0pt}{lime},
  before skip=10pt,
  after skip=10pt,
  sharp corners,
  left=5pt,
  right=5pt,
  breakable,
}